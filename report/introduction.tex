\section{Introduction}
This report is an extention of our bachelor thesis (Barten 2018), that aimed to detected slums from satellite images of Bangalore using the features Histogram of Oriented Gradients and Line Support Regions together with a third feature that used the Road intersection Density. We confirmed that these features were able to detect slums in the provided satellite images, although the performance of the approach was limited. 

In both the thesis of Filtenborg and us, we encountered problems with the two class division of the image. The image was divised into slum and everything else. The large area that is not slum has a high visual variance as this included a lot of vegetation, buildings and soil. We therefore explore the impacts 3 classes instead of 2. Because large parts of the test image contain vegetation, we add a third class that represents the vegetation. This results in three classes, slum, building and vegetation. We believe that the addition of another class would improve the detection of slums.



