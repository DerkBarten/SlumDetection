\section{Introduction}
This report is an extension of our thesis which aimed to detected slums from satellite images of Bangalore using the features Histogram of Oriented Gradients and Line Support Regions together with a third feature that used the Road intersection Density \cite{derk}. We concluded that these features were able to detect slums in the provided satellite images, although the performance of the approach was limited. 

In both our thesis and in the work of Filtenborg \cite{max}, we encountered problems with the two-class division of the image. The image was divided into a slum class and a class containing everything else. The large area that is not slum has a high visual variance because of the different land covers it contains, such as vegetation, buildings, and soil. Therefore, this report explores the effect of the addition of another class. Because large parts of the test image contain vegetation, the third class represents the vegetation. The second class is used as a class of leftovers that includes everything that is not vegetation, soil or slum. This mask generally corresponds with buildings and is therefore called the building mask. We expect that the exclusion of the vegetation into a separate class will improve the classification performance of the slum class.