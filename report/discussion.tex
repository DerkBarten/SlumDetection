\section{Discussion and Conclusion}
There is a clear improvement when using three classes instead of two, as displayed in Figure \ref{fig:comp}. Because both the MCC and F1 score use the precision and recall, it is likely that the increase in recall caused the increase of these two measures. With three classes for the third section, it seems that it is able to maintain a high recall when the precision is high. In contrast, the two-class classification has a significantly lower recall while the precision is equal to that of the three class classification. The improved performance results in relatively accurate predictions with very little noise, with the exception of the first section, as visualized in \ref{fig:three_classes}a.

Interestingly enough, Figure \ref{fig:three_classes}c seems to detects traffic as areas as slums. Upon further investigation, the morphology of cars on the road seems to match that of slums. On satellite images, cars resemble small scattered, disorganized, objects, thus both in appearance and in texture similar to slums. This effect is perhaps amplified by the fact that the features that are used are largely based on texture. Furthermore, the other two images that are used for training do not contain large amounts of traffic, therefore the algorithm has not learned that these vehicles do not fall under the slum class.

The experiment with the different block sizes suggests that the recall increases while the precision drops, displayed in Figure \ref{fig:block}. An increasing block size should cover more of the ground truth, but therefore also detects areas that are not slums. This is well illustrated in Figure \ref{fig:block}c, where almost all slum areas are covered by the prediction but also a lot of noise is introduced. This is clearly a balancing act between the precision and recall that will depend on the specific goals the slum detection is used for. Although Figure \ref{fig:block}b has a substantial difference in recall and precision, it still scores a comparatively same MCC as Figure \ref{fig:block}a. Another motivation for higher block size might be the lower computational cost. On an increasingly larger scale, it becomes more important that the feature calculation and the classification falls within a certain time window.

The threshold parameter does not seem to be a balancing act like the block size. Both recall and precision increase with the threshold value. A higher threshold value removes blocks that are only partly covered by the slum mask. The slum mask is often generous in its coverage, as it often includes an area around the actual slum. Higher threshold values might trim the shapes of slums, effectively removing noise from the slum class and improving performance.

The experiment we performed with oversampling did not seem to produce improvements in classification performance. We encountered that, for both two times and four times slum oversampling, the resulting performance was exactly the same. It is uncertain whether this is valid or caused by a technical fault. The case where the classes have equal numbers has significantly lower performance than without oversampling. This is caused by a very high recall with low precision. We expect the recall to increase with oversampling because the occurrence of slum would become more common, therefore the learning algorithm would more often tend to classify an area as slum compared to no oversampling. This increase in recall reduces the precision significantly and therefore reducing the MCC as well.

Besides the negative effects of oversampling on three-class classification, the exclusion of oversampling might not be fair to the two-class classification. This is unfair because oversampling increases the two-class performance as we and Filtenborg experienced in our theses. Because we only want to compare the impact of the third class, we excluded from the comparison. However, when applied in practice, it might be more beneficial to create a comparison to discover whether to create a third class or use oversampling on two-class classification instead.





