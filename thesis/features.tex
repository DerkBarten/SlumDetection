\section{Conventional Feature Extraction Methods}
Researchers in this field of study often use a combination of features in the detection of informal settlements. Our research uses features that performed well in previous research;  the Histogram of Oriented Gradients (HoG) and Line Support Region (LSR) features. Both HoG and LSR are implemented in a Python library, called \textit{Spfeas}, which is based on the research of Graesser \textit{et al.} \cite{graesser2012image}. Alongside these features, we design a new feature that is based on the difference in the distribution of road intersections between formal and informal areas.

\begin{figure}
    \centering
    \includegraphics[width=0.5\textwidth]{images/block_scale}
    \caption{An example of block size and scale}
    \label{fig:block_scale}
\end{figure}


\subsection{Terminology}
The paper of Graesser \textit{et al.} divides the image into small blocks instead of pixels when extracting features from the satellite images. The features are calculated for each block of pixels instead of each pixel individually, which significantly lowers the computational load as the extraction methods can be computationally quite expensive. The dimensions of the blocks are referred to as the \textit{block size},  which is 20 by 20 pixels in the paper of Graesser \textit{et al}. In our research, various block sizes will be evaluated for their effect on the classification performance. Besides \textit{block size}, another important parameter in the paper is the \textit{scale} of a feature. The \textit{scale} specifies an n by n block around the pixel block of which the features are calculated.

Figure \ref{fig:block_scale} is an example of how the size of the block and the scale are used in the calculation of the features from an image. In this example, the size of the image is 120 by 120 pixels and is divided into blocks of the 20 by 20 pixels each, which is the block size. In this example, the scale is the area around each of 80 by 80 pixels. For every block in the image, starting from the top left to the bottom right, the features are calculated over the area covered by the scale. To clarify, if we extract $n$ features from the image in the example, the dimensions of the feature vector would be (6 x 6 x $n$). The block size essentially controls the resolution of the calculated feature vector with an increased block size causing a decreased resolution and the same in reverse. In practice, the block size is almost never a multiple of the image size. In the border regions of the image using a scale larger than the block size, the outer blocks where the scale falls off the edge of the image are discarded. This results in a smaller feature vector than merely dividing the size of the image by the block size.

\begin{figure}[h]
    \centering
    $\mathlarger{\mathlarger{\mathlarger{   p_x = \ceil*{\frac{i_x}{b_x}} - \ceil*{\frac{i_x - s_x - b_x}{b_x}} }}}$
    \caption{Definition of the padding calculation in the $x$-direction}
    \label{padding}
\end{figure}

Figure \ref{padding} shows displays the equation to calculate how many blocks will be removed as padding $p$ for a particular image $i$ with the combination of scale $s$ and block size $b$. We displayed the function for the calculation of the padding in the $x$-direction, although this will be different for the $y$-direction if the image, scale or block size is not of equal width and length. In our case, only the images will have a different width and length; the scales and block sizes will always be $n$x$n$; thus we will be referring two these parameters with a single value $n$ instead of $n$x$n$

We used this equation to predict the dimensions of the features that the Python library would produce after we supplied the input image and the accompanying parameters. We need this prediction to reshape the ground truth into the same shape as the produced features. If the shape of the ground truth does not match the size of the feature, we cannot correctly label the values from the feature vector as either belonging to the formal or informal class and produce the dataset for training and testing for the classification algorithms.

\subsection{Histogram of Oriented Gradients}

Although the patent application describing the Histogram of Oriented Gradients was submitted in 1986, the approach only became widespread in 2005, after a paper used this method to detect humans on images \cite{dalal2005histograms}. The Histogram of Oriented gradients creates a histogram for every block in the image where the histogram contains the gradient orientation of the pixels in the block and surrounding area, as determined by the scale and block size parameter. In the case of the detection of humans, for example, the visual differences between humans and the background manifests itself in the gradient orientations of the image, which the Histogram of Oriented Gradients can capture. The difference in gradient orientation, therefore, enables objects with distinct visual characteristics, such as humans, to be detected from images.

Beyond the detection of humans, a paper from 2003 showed that this approach can also be used to detect man-made structures in photographs \cite{kumar2003man}, which used images of buildings surrounded by vegetation. The Histogram of Oriented Gradients method described in the paper of Graesser \textit{et al.} is based on this paper, although they used satellite images instead of regular photographs from nature. As in the method for the detection of humans with the Histogram of Oriented Gradients, the paper from Graesser \textit{et al.} captures characteristics of a particular class, which are, in this case, the characteristics of informal neighborhoods. In case of slums, these characteristics are the diverse orientations of gradients in a slum area due to diverse building orientations compared to formal structures. In contrast to slums, formal buildings are often placed in a regular pattern with consistent orientation.


The Python library we used extracts different features from the Histogram of Oriented Gradients than the ones described in the paper. The paper uses the first two central moments together with three orientation features, but they are ill-described; the library instead uses four central moments together with a maximum. Because we use the implementation from the library, we will only discuss the features that are used in the library instead of the paper. 

The four central moments that are used form a set of values that characterize a probability distribution relative to the mean of the distribution. These characteristics of a distribution, better known as the mean, variance, skew and kurtosis, are defined using the formula displayed in \ref{central_moments} \cite{grimmett2001probability}. In this formula, $n$ is the order of the central moment, $\mu_{n}$ is the $n$th central moment and $\mu$ without a subscript is the mean of the distribution on which the central moments are based. The zeroth and first central moment are trivial because $n=0$ and $n=1$ will always result in 1 and 0 respectively. Instead of using the trivial first central moment, the library uses the regular mean instead. The other three central moments, variance, skew and kurtosis are calculated using the formula in \ref{central_moments} for $n=2$, $n=3$, and $n=4$, respectively. Besides the central moments, the fifth feature is a maximum although, due to lack of documentation, it is unclear what this maximum refers to exactly.

\begin{figure}[h]
    \centering
    $\mathlarger{\mathlarger{\mu_{n} = \int_{-\infty}^{+\infty} (x-\mu)^nf(x)dx }} $
    \caption{The definition of the central moments}
    \label{central_moments}
\end{figure}

Instead of using a single scale for feature calculation, the paper of Graesser \textit{et al.} uses octaves of three scales, meaning that the size of the scale doubles for every scale. To illustrate, the scale octaves used for both the Histogram of Gradients and the Line Support Regions, is 50, 100, and 200. To summarize: in the paper, a single HoG feature vector contains five values, since the feature is performed for three different scales, the total values in the feature vector results in 15. Although it is not mentioned in the paper, the library calculates the features for the different color bands as well, resulting in 45 values instead of 15, which implies that in the paper only a single color band was used.

According to the results presented in the paper, these 15 features could produce an accuracy of 65 to 75 percent. However, this feature was applied to specific image regions where the visual difference between formal and informal was substantial. It is therefore debatable whether this performance is to be attributed to the Histogram of Oriented Gradients or the specific contents of the image. Besides, the morphology of the slums in their area of study is different from the slums in Bangalore, making a comparison difficult.


\subsection{Line Support Region Features}

The Line Support Regions method was initially used for the detection of straight lines in photographs \cite{burns1986extracting}. As with the Histogram of Oriented Gradients, this method is a spatial feature and uses gradient orientation to characterize parts in the image, in this case, straight lines. This approach groups pixels together with similar gradient orientation based on the fact that straight lines are in essence regions of pixels with similar gradients. 

This approach was shown to be suited for land use classification \cite{unsalan2004classifying} \cite{unsalan2006gradient}, and has been used in slum and informal region detection since \cite{graesser2012image} \cite{accra} \cite{colombo}. LSR characterizes neighborhoods using the lines that inhabit the area, which often corresponds to the contours of buildings. In formal neighborhoods, these lines are often relatively long since the formal structures tend to be bigger than informal structures. Furthermore, according to the paper \cite{unsalan2004classifying}, line contrast can be used as well to differentiate between land use, as developed areas tend to have high contrast as opposed to low contrast in underdeveloped areas. This difference in contrast can, for instance, be caused by the presence of asphalt, shining roof material and vegetation, which tends to lack in underdeveloped regions. In the paper, this approach was used to differentiate between urban areas and rural areas, although this is now used as well to differentiate between region types within an urban area.

LSR is implemented in Spfeas in accordance with the paper of Graesser \textit{et al.} \cite{graesser2012image}. The paper uses the line length entropy, mean, and entropy of line contrasts as statistical features from the Line Support Regions, although the paper does not describe this process detail. The paper to which Graesser \textit{et al.} refers to, describes the process of calculating these statistical features using three coefficients produced by a Fourier Transformation \cite{unsalan2004classifying}. These coefficients are $(\alpha_{-1}, \beta_{-1})$, $(\alpha_{0}, \beta_{0})$, and $(\alpha_{1}, \beta_{1})$ of which the derivation can be found in the paper.

\begin{figure}[h]
    \centering
    $$\mathlarger{\mathlarger{ \mu_{xy} = (\alpha_{0}, \beta_{0}) }}$$
    $$\mathlarger{\mathlarger{ l = 2\left[\sqrt{\alpha_{1}^2 + \beta_{1}^2} + \sqrt{\alpha_{-1}^2 + \beta_{-1}^2}\right] }}$$
    $$\mathlarger{\mathlarger{ \theta = \frac{\arctan(\frac{\beta_{1}}{\alpha_{1}}) + \arctan(\frac{\beta_{-1}}{\alpha_{-1}})}{2} }}$$
    \caption{Definition of the center of mass $\mu_{xy}$, the line length $l$ and orientation $\theta$}
    \label{line_def}
\end{figure}

Using these coefficients, we can characterize the lines detected within the scale of a block by using the center of mass $\mu_{xy}$, the line length $l$ and orientation $\theta$ with the formula displayed in Figure \ref{line_def}. The most straightforward statistical feature is the mean, which is calculated by taking the mean of all $l$ values of the detected line support regions. The other two features, the entropy of the line length and the line contrast,  are calculated using the entropy formula in Figure \ref{line_entropy_def}. 

\begin{figure}[h]
    
    $$ \mathlarger{\mathlarger{ E = -\sum_{i=1}^{N} \left[ h(i) \log_2(h(i))\right] }}$$
    \caption{Definition of the entropy function}
    \label{line_entropy_def}
\end{figure}

In the creation of the line length entropy, the length of the lines in the area is collected in a histogram with bins representing the different length of the lines. The paper\cite{unsalan2004classifying} describes a total of 37 bins ranging from 5 to 150 pixels. Regarding the formula in Figure \ref{line_entropy_def}, using the histogram $h$ with $N$ bins, we can calculate the entropy of the line length $E_l$.

The line contrasts are calculated by finding the maximum gradient for every line support region. These maximum gradients are, as the line contrasts calculation, collected in a histogram of 31 bins, ranging from 5 to 3000. A more detailed description including the formula for the maximum feature selection can be found in the paper \cite{unsalan2004classifying}. The histogram is used by the entropy function defined in Figure \ref{line_entropy_def} and results in the line contrast entropy $E_c$.

The paper by Graesser \textit{et al.} uses the same octave scale and block size used for HoG, resulting in a total of 9 features. Again, since we use three color bands, the number of features used in our implementation is multiplied by three relative to the paper. Using the LSR features, the paper achieved an accuracy of 60 to 75 percent.



