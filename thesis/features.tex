\section{Conventional Feature Extraction Methods}

In the classification of areas in satellite images, it is common for studies to use
a combination of features on which classification can be performed.
Our research will use methods for feature extraction that performed well in
previous research \cite{graesser2012image}. The methods that are used, are: the Histogram
of Oriented Gradients, Line Support Region features and Linear Feature
Distribution. Both the Histogram of Oriented Gradients and Line Support Regions
are already implemented in a framework created by Jordan Graesser. This
framework is a Python library, called spfeas, that is based on the
research Graesser et al conducted on the classification of formal and informal
settlements \cite{graesser2012image}. In addition to the features provided by
spfeas, this research will implement a new feature that measures the density of
road intersections in a given area, with the belief that area's with many
intersections are less likely to be informal.

\subsection{Histogram of Oriented Gradients}
% designed for only settlement detection
% as supporting feature
% theory part
% however, Graesser does work for differentiating
% picture for explaination?


The Histogram of Oriented Gradients (HoG) was originally used for the detection of
man-made structures in images. The features retrieved from the histogram were
able to differentiate between natural and man-made sections in an image. This
approach of man-made structure detection relies on the difference in the
distrubution of gradients in a particular region of the image. Because man-made
structures are often consists of straight lines and regular forms, this
contrasts the irregular forms that are naturally formed. This difference in
appearance translates to a difference in the distribution of gradients in
man-made structures and nature.

In satellite images, the difference of the Histogram of Oriented Gradients
between nature and human settlements is quite large. Eventhough HoG was
originallly used for differentiating between nature and man-made structures,
the paper of Graesser et al was able to succesfully to use HoG between formal
and informal neighborhoods. The paper uses the disparate spatial distribution of
buildings in formal and informal regions. The
placement of houses in formal regions are often placed in a regular pattern
with fixed orientations. Informal areas, in contrast, are usually constructed
without a design or regular pattern. This difference in regularity is
a characteristic that can be used for classification of formal or informal
settlements.

The order of regularity of an area can be captured using the number of
orientations in which buildings are constructed. Few distinct orientations suggest
a formal settlements while many orientations suggest informal. The orientations
of buildings are calculated using the gradients returned by the application of
a binary filter on the image. The gradients are quantized into a set of
orientations or bins. This results in a histogram commonly named the Histogram
of Oriented Gradients. High peaks in the HoG correspond to a multitude of
similar gradients, thus uniformity in the image. Since uniformity is linked
to formal settlements, high peaks in the histogram correlate with formal
settlements. The informal settlements, on the other hand, are characterized by
the absence of peaks in the HoG due to the multitude of different gradients
caused by the irregular placement of buildings.

Applied in practice, the paper from Graesser et al extracts five
characteristics from the HoG: two types of mean, variance, skew, and
kurtosis. These properties of the histogram are used as features to
describe the histogram and differentiate between formal and informal regions.
These 5 features will be calculated for every color band of the image,
resulting in a grand total of 15 features for the Histogram of Gradients.
According to the results presented in the paper, these 15 features alone could
result in an accuracy of 65 to 75 \%.

\subsection{Line Support Region Features}

The formality of regions can be for a part be determined by observing the
spatial distribution of neighborhoods. In the same manner as HoG characterized
areas by the orientation their buildings, neighborhoods, alternatively, can be
characterized by the size of their buildings. Informal settlements generally
lack the precence of large buildings in contrast with formal regions. The size
of constructions can be characterised using the Line Support Region (LSR)
features \cite{unsalan2004classifying}. Likewise to HoG, LSR utilizes gradients
calulated from remote sensing imagery. LSR uses the fact that straight lines
have uniform gradients. In practice, natural photographs hardly contain
perfectly straight lines, which results in similar but non uniform gradients of
lines. Therefore, LSR uses groups similar gradients to represent a line in
images to detect semi straight lines in images \cite{burns1986extracting}.

The LSR is implemented in spfeas in accordance to the paper of Graesser et al
\cite{graesser2012image}. The paper uses three statistical features extracted
from the LSR, these are line length entropy, mean, and entropy of line
contrasts. Together with three color bands, this brings the total to nine
features for a single scale.
