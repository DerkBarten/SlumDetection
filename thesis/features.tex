\section{Conventional Feature Extraction Methods}
Researchers in this field of study often use a combination of features in the detection of informal settlements. Our research uses features that performed well in previous research. The use two methods for extracting features from the satellite images; these are the Histogram of Oriented Gradients (HoG) and Line Support Region (LSR) features. Both HoG and LSR are implemented in a Python library, called spfeas, which is based on the research of Graesser \textit{et al.} \cite{graesser2012image}. Along side these features, we design a new feature that is based on the difference 
in distribution of road intersections between formal and informal areas. This is based on the belief that there is an observable difference between formal and informal areas in this aspect.

\subsection{Terminology}
The paper of Graesser \textit{et al.} divides the image in small blocks instead of pixels when extracting features from the satellite images. The features are calculated for each block of pixels instead of each pixel itself, which significantly lowers the computational load as the extraction methods can be computationally quite expensive. The dimensions of the blocks are referred to as the \textit{block size},  which is 20 by 20 pixels in the paper of Graesser \textit{et al.}. In our research, various block sizes will evaluated for the effect on the classification performance.
Besides \textit{block size} , another important parameter in the paper is the \textit{scale} of a feature. The \textit{scale} specifies the pixels surrounding the block that should be included in the calculation of the feature, allowing blocks to be
spatially connected to the surrounding areas, which may increase performance.

\subsection{Histogram of Oriented Gradients}

Although the pattent application describing the Histogram of Oriented Gradients was submitted in 1986, the approach only became popular in 2005, after a paper used this method to detect humans on images \cite{dalal2005histograms}. The Histogram of Oriented gradients creates a histogram for every block in the image, were the histogram contains the gradient orientation of the pixels in the block and surrounding area, as determined by the scale and block size parameter. In case of the detection of humans, for example, the visual visual differences between humans and the background manifests itself in the gradient orientations of the image, which the Histogram of Oriented Gradients is able to capture. The difference in gradient orientation therefore enables objects with distinct visual characteristics, such as humans, to be detected from images.

Beyond the detection of humans, a paper from 2003 showed that this approach can also be used to detect man made structures in photographs \cite{kumar2003man}, which used images of buildings surrounded by vegetation. The Histogram of Oriented Gradients method described in the paper of Graesser \textit{et al.} is based on this paper, although they used satellite images instead of regular photographs from nature. As in the method for the detection of humans with the Histogram of Oriented Gradients, the paper from Graesser \textit{et al.} captures characteristics of a certain class, which are, in this case, the characteristics of informal neighborhoods. In case of slums, these characteristics are the diverse orientations of gradients in a slum area due to diverse building orientations compared to formal structures. In contrast to slums, formal buildings are often placed in a regular pattern with consistent orientation.

Graesser \textit{et al} extracts five features from the Histogram of Oriented Gradients: first and second order of the heaved central shift moment, variance, skew, and kurtosis. These properties of the histogram are used as features to describe the histogram. These 5 features are calculated for every color band of the rgb image, resulting in a total of 15 features for the Histogram of Gradients. According to the results presented in the paper, these 15 features could produce an accuracy of 65 to 75 percent, although it must be noted that this feature was applied to specific regions where the visual difference between formal and informal was substantial. Furthermore, the morphology of the slums in their area of study is different that the slums in Bangalore.


\subsection{Line Support Region Features}

The Line Support Regions method was originally used for the detection of straight lines in photographs \cite{burns1986extracting}. As with the Histogram of Oriented Gradients, this method is a spatial feature and uses gradient orientation to characterize parts in the image, in this case, straight lines. This approach groups pixels together with similar gradient orientation based on the fact that straight lines are in essence regions of pixels with identical gradients. 

This approach was shown to be suited for land use classification \cite{unsalan2004classifying} \cite{unsalan2006gradient}, and has been used in slum and informal region detection since \cite{graesser2012image} \cite{accra} \cite{colombo}. LSR characterizes neighborhoods in the their lines, which often correspond to the contours of buildings. In formal neighborhoods, these lines are often relatively long since the formal structures tend to be bigger in size than informal structures. This contrast in the properties of lines in certain regions can characterize the region as either formal or informal.

As with the Histogram of Oriented Gradients, LSR is implemented in spfeas in accordance to the paper of Graesser \textit{et al.} \cite{graesser2012image}. The paper uses three statistical features extracted from the LSR, these are: line length entropy, mean, and entropy of line contrasts. Together with three color bands, this brings the total to nine features for a single scale. Using the LSR features for the scales 50, 100, and 200, the paper achieved an accuracy of 60 to 75 percent.

%In the same manner as HoG characterized
%areas by the orientation their buildings, neighborhoods, alternatively, can be
%characterized by the size of their buildings. Informal settlements generally
%lack the presence of large buildings in contrast with formal regions. The size
%of constructions can be characterized using the Line Support Region (LSR)
%features \cite{unsalan2004classifying}. Likewise to HoG, LSR utilizes gradients
%calculated from remote sensing imagery. LSR uses the fact that straight lines
%have uniform gradients. In practice, natural photographs hardly contain
%perfectly straight lines, which results in similar but non uniform gradients of
%lines. Therefore, LSR uses groups similar gradients to represent a line in
%images to detect semi straight lines in images \cite{burns1986extracting}.


