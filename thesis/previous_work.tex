\section{Previous Work}

There have been multiple studies concerning the detection of informal
settlements using remote sensing in the last decade \cite{kuffer2016slums}.
Informal settlements in the proximity of cities have been studied throughout
the globe using remote sensing imagery, e.g. Colombo \cite{colombo},
Johannesburg \cite{williams2016automatic}, Accra \cite{accra}, Mumbai
\cite{mumbai}, and Hyderabad \cite{hyderabad}. There are numerous
characteristics that are suitable to differentiate between formal and informal
settlements. This could, for example,  be the number vegetation, the width of
the roads, the size and orientations of dwellings \cite{owen2013approach}.

In general, there is a shift towards the usage of objects for the
classification of geographical features. This is broadly known as Geographic
Object-based imaga analysis or GEOBIA \cite{hay2008geographic}.Due to the
increase in availability of remote sensing high resolution imagery, it became
feasible to distinguish individual objects on a large scale.  This allows for
the interpretation of geographical imagery with individual objects at its
basis.  As a result, the characterization of regions may be based on the type
of objects that inhabit it.  To illustrate, a study classified rooftop objects
from aerial imagery of an informal settlements in Johannesburg to gather
demographic information as an alternative to traditional ground based survey
\cite{williams2016automatic}.

When classifying entire neighborhoods, it is common for studies to use
a combination of features for the classification between formal and informal.
Our research will use methods for feature extraction that performed well in
previous research \cite{graesser2012image}. The methods used are: the Histogram
of Oriented Gradients, Line Support Region features and Linear Feature
Distribution.


