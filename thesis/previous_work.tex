\section{Previous Work}

There have been multiple studies concerning the detection of informal
settlements using remote sensing in the last decade \cite{kuffer2016slums}.
Informal settlements in the proximity of cities have been studied throughout
the globe using remote sensing imagery, e.g. Colombo \cite{colombo},
Johannesburg \cite{williams2016automatic}, Accra \cite{accra}, Mumbai
\cite{mumbai}, and Hyderabad \cite{hyderabad}. There are numerous
characteristics that are suitable to differentiate between formal and informal
settlements. This could, for example,  be the number vegetation, the width of
the roads, the size and orientations of dwellings \cite{owen2013approach}.

In general, there is a shift towards the usage of objects for the
classification of geographical features. This is broadly known as Geographic
Object-based imaga analysis or GEOBIA \cite{hay2008geographic}.Due to the
increase in availability of remote sensing high resolution imagery, it became
feasible to distinguish individual objects on a large scale.  This allows for
the interpretation of geographical imagery with individual objects at its
basis.  As a result, the characterization of regions may be based on the type
of objects that inhabit it.  To illustrate, a study classified rooftop objects
from aerial imagery of an informal settlements in Johannesburg to gather
demographic information as an alternative to traditional ground based survey
\cite{williams2016automatic}.


Our study continues partly on the work performed by Graesser et al.  A study
from 2012 conducted by Graesser et al  investigated similar
problem, the characterization of formal and informal neighborhoods.  This study
uses a collection of different features to capture the difference in
characteristics between formal and informal neighborhoods in urban
environments. Our research will evaluate if the results obtained by Graesser et
al translates well to different cities with imagery from a different
satellite. Beyond the replication of previous research, the features used by
Graesser will be extended by an additional feature. This feature will be
compared to the existing features in various measures. The features combined
will be used by a set of different classification algorithms to both assess the
performance of the feature set as well as the evaluation of various
classification algorithms.


% difference everything not informal automatically fromal
