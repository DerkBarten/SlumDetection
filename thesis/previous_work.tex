\section{Previous Work}

There have been quite a few studies concerning the detection of informal
settlements using remote sensing in the last decade \cite{kuffer2016slums}.
Informal settlements inside and in the proximity of cities have been studied
from all around the world using remote sensing imagery, for instance Colombo
\cite{colombo}, Johannesburg \cite{williams2016automatic}, Accra \cite{accra},
Mumbai \cite{mumbai}, and Hyderabad \cite{hyderabad}. There are numerous
characteristics that can be used to differentiate between formal and informal
settlements. This could be the number vegetation, the width of the roads, the
size and orientations of dwellings \cite{owen2013approach}.

In general, there is a shift towards the usage of objects to classify geographical
features, also known as Geographic Object-based imaga analysis or GEOBIA
\cite{hay2008geographic}.Due to the increase in availability of remote sensing
high resolution imagery, it became possible to distinguish individual objects
on a large scale.  This allows for the
interpretation of geographical imagery with individual objects at its basis.
As a result,  this enable the characterization of regions based on the type of objects
that inhabit it.  As and example, a study classified rooftop objects from
aerial imagery of an informal settlements in Johannesburg to gather demographic
information as an alternative to traditional survey
\cite{williams2016automatic}.

When classifying entire neighborhoods, it is common for studies to use a combination
of characteristic that are used in the classification between formal and
informal \cite{graesser2012image}. As previously mentioned, the orientation of
buildings is used as a way of discriminating between formal and informal
settlements. A technique used to qualify the orientation is the Histogram of
Oriented Gradients (HoG)\cite{kumar2006discriminative}. This algorithm iterates over blocks of pixels in a given
image and calculate the gradient. The gradients are binned into
the bin that most closely represents their orientation. This effectively
translates the infinitely many orientations of the gradients to just a few.
Structured objects, such as the walls, are characterized by peaks in a few
bins because the gradients in the block will be very alike

