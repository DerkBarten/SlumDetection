\textbf{\textit{Abstract} - Many cities around the globe encounter problems with slum development. The detection slums using remote-sensed images provides a low-cost solution to gain insight in the location and dynamics of slums in a city; this enables governments to support these slums and provide basic services. In satellite images, the distinct morphology of slums that separates them from other building types has been characterized by numerous features. This thesis aims to detect slums in satellite images from Bangalore, India, using the well-known Histogram of Oriented Gradients, Line Support Regions together with a new feature called the Road Intersection Density. These features are used by the classifiers AdaBoost, Gradient boosting and Random Forrest, among others to detect slums in the satellite image. Using the Histogram of Oriented Gradients and Gradient boosting, we were able to detect slums with a Matthews coefficient of 0.17. Our newly designed feature did not reach the same performance as the conventional Histogram of Oriented Gradients and Line Support Regions. }