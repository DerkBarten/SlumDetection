
\section{Feature Evaluation}

In order produce a functioning classifier, the predictive value of the features
must be evaluated. This assesses wether a certain feature is useful in the
discrimination between formal and informal. The evaluation requires the
knowledge of what parts in a satellite image are formal and informal to create
a distinction between the two classes. This knowledge is represented in
a ground truth, this is a mask covering the satellite image indicating the
informal regions.

The mask is used to divide each block of pixels into the two classes, formal
and informal. Each block has a vector of values, also called features,
characterizing that particular block. Because the blocks are grouped together,
the features associated with the same class are grouped together as well. This
allows for the analysis of the distribution of the features in both classes. If
the distribution of a certain feature varies quite distinctly between the two
classes, this indicates as a suitable candidate for classification of informal
regions.

The groundtruth mask is constructed using a vector file containing the
boundaries of informal areas. The boundary file is rasterized and applied on
top of the satellite image, creating a mask of the informal settlements.
Because classification and feature calculation is performed on blocks instead
of pixels, the pixel based mask transformed into a block based mask. In this
mask, every pixel represents a block where the zero represents formal and one
represents informal. This effectively creates a ground truth.  

The predictive value of a feature can be visualized using a boxplot of the two
classes. When a feature is distinctive on its own, the distribution of the
values in the two classes will be different. Eventhough there might not be
a significant difference between two distribution, by using multiple features
in combination, the grand total might more distinctive that the sum of its
parts. 



