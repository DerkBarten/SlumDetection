\subsection{Histogram of Oriented Gradients}
The spatial distribution of buildings in formal and informal settlements varies
significantly. The placement of houses in formal settlements are often placed
in a regular pattern with fixed orientations. Informal settlements, in
contrast, usually constructed without a design or regular pattern. This
difference in regularity is a characteristic that will be used
for classification of formal or informal settlements.

The order of regularity of an area can be captured using the number of
orientations in which buildings are constructed. Few orientations suggest
a formal settlements while many orientations suggest informal. The orientations
of buildings are calculated using the gradients returned by the application of
a sobel filter on the image. The gradients are quantized into a set of
orientations or bins. This results in a histogram commonly named the Histogram
of Oriented Gradients (HoG). High peaks the HoG correspond to a multitude of
similar gradients, thus uniformity in the image. Since uniformity is linked
to formal settlements, high peaks in the histogram correlate with formal
settlements. The informal settlements, on the other hand, are characterized by
the absence of peaks in the HoG due to the multitude of different gradients
caused by the irregular placement of buildings.

The HoG itself cannot reasonably be used as a feature, therefore the properties
of its distribution are used instead. One property of the HoG that well defines the
difference between formal and informal are the heaved central-shift moments.
The heaved central-shift moments are a method for the calculation of the
weighted mean of a histogram for any order.  The orders correspond to the
impact that high peaks in the histogram have on the mean. 


Histogram of Oriented Gradients
that characterize a certain area.


% praktisch hoe het hier gedaan is

