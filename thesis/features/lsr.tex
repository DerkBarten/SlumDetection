\subsection{Line Support Region Features}

The formality of regions can be for a part be determined by observing the
spatial distribution of neighborhoods. In the same manner as HoG characterized
areas by the orientation their buildings, neighborhoods, alternatively, can be
characterized by the size of their buildings. Informal settlements generally
lack the precence of large buildings in contrast with formal regions. The size
of constructions can be characterised using the Line Support Region (LSR)
features \cite{unsalan2004classifying}. Likewise to HoG, LSR utilizes gradients
calulated from remote sensing imagery. LSR uses the fact that straight lines
have uniform gradients. In practice, natural photographs hardly contain
perfectly straight lines, which results in similar but non uniform gradients of
lines. Therefore, LSR uses groups similar gradients to represent a line in
images to detect semi straight lines in images \cite{burns1986extracting}.

The LSR is implemented in spfeas in accordance to the paper of Graesser et al
\cite{graesser2012image}. The paper uses three statistical features extracted
from the LSR, these are line length entropy, mean, and entropy of line
contrasts. Together with three color bands, this brings the total to nine
features for a single scale.
