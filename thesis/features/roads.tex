\subsection{Intersection Density}

% how density could indicate formal
% the estimated accuracy?
% add picture to explain
% how intersections found in images
% ^ theoretical background
%

As a novel approach for the distinction between formal and informal regions, we
propose the density of intersections as a metric. This is on the belief that
area's with a dense road network, resulting in many intersections, are well
developed, thus indicating a formal region. Informal regions, on the other
hand, would be characterized by a dip in the density of intersections.  

This metric is constructed using road extraction techniques. The area of road
extraction has developed over the years, starting from 1970. A straightforward
method for road extraction is the use of edge detection. Because roads have
distinct properties due to material, design and function, the appearance of
roads are often quite in contrast to the environment. For example, on satellite
images, the homogenious black color and the regular straight lines make
roads easily identifiable from the suroundings for the human eye. Edge
detection uses the sharp contrast between a road and the roadside to identify
the road.

Since, in our case, were only interested in the positions of the intersection
and not the entirety of the road. 
