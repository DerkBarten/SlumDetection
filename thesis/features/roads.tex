\subsection{Road Intersection Density}

% how density could indicate formal
% the estimated accuracy?
% add picture to explain
% how intersections found in images
% ^ theoretical background
%

As a novel approach for the distinction between formal and informal regions, we
propose the density of intersections as a metric. This is on the belief that
area's with a dense road network, resulting in many intersections, are well
developed, thus indicating a formal region. Informal regions, on the other
hand, would be characterized by a dip in the density of intersections.  

This metric is constructed using road extraction techniques. The area of road
extraction has developed over the years, starting from 1970. A straightforward
method for road extraction is the use of edge detection. Because roads have
distinct properties due to material, design and function, the appearance of
roads are often quite in contrast to the environment. For example, on satellite
images, the homogenious black color and the regular straight lines make
roads easily identifiable from the suroundings for the human eye. Edge
detection uses the sharp contrast between a road and the roadside to identify
the road.

Besides tradional image processing methods, another promising approach in the
detection of road networks from satellite images are Neural Networks
\cite{mangala2011extraction} \cite{mokhtarzade2007road}. A study from 2017 was
able to extract both the road network together with buildings with high
accuracy using a Convolutional Neural Network \cite{alshehhi2017simultaneous}.

We used a seperate implementation of the convolutional neural network since the
research paper did not include the software used \cite{airs}. This
implementation includes a set of images, that could be used for the training
and validation of the neural network \cite{MnihThesis}. After the training of
the network on the provided images, was used to extract the roads from our own
satellite images.  Unfortunately, the mask of the road network was erroneous as
it did not represent the road network in the provided image. The suspected
cause of this failure is the difference in the training data to our satellite
imagery data.

To illustrate the difference, the trainingset contained satellite images
obtained from partly rural area's of the state of Massachusetts in the United
States while the data used in our research is from Bangalore in India which is
mostly urban.  As a result, the geographical features and the road systems are
quite different in the two area's. This could cause the network not to
recognize the roads in the images from Bangalore. The difference in resolutions
of the two image sets could be another cause. The images of Massachusets were
of a lower resolution than the images of Bangalore which could have hindered
the neural network in the correct classification of the roads, although this is
ungrounded speculation.

% TODO: cite Otsu
As a alternative to neural network, conventional image processes were used. The
image is subjected to a number of operations that will extract the road network
from the image. The first operation is the transformation of the RGB satellite
image to grayscale values. This is done in preparation for Otsu's method for 
threshold, which effectively separates buildings from roads. The resulting
image, although quite crude, is a mask of the road network in the satellite image.   

% TODO: hough transform
% perhaps expand otsu more

% Use opencv canny stuff with hough lines

