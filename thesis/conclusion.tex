\section{Conclusion}

Our approach was able to successfully detect slums on the provided images using the features described by Graesser \textit{et al.}, albeit with limited performance. We have experimented with different sets of parameters to find the best combination, which produced Matthews coefficient of 0.17. We discovered that the Histogram of Gradients and Gradient boosting achieves the highest performance in general. Our proposed method for the detection of slums using road intersection detection did not seem to produce useful results, although the intersection detection might be used for different problems. The effect of different scales is still debatable, and increased block size does not seem to impact the performance of the features. All oversampling methods we have experimented with resulted in similar performance, which is significantly higher than no oversampling at all. 

This thesis was written in parallel with a colleague who worked on the same problem and with different features and methods. The features presented in this thesis in combination with their thesis may improve performance further.


% reflection?


% Summary of results, which parameters worked well
% Reached goal?






%\section{Future Work}
%As you now know, we performed the experiments on three test images. These images are only a small part of the whole image. This was done to reduce computational load and decrease the class imbalance. We have not classified using the whole image, although this could be the next logical step.
%
%would be classification of the whole image as a test set, and the three sections as training set. 