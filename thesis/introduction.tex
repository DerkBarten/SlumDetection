\section{Introduction}

Growth of informal settlements is a common issue faced by nations globally.
These settlements are not limited to slums in undeveloped countries but also
include refugee camps in the developed world. Informal settlements might be
difficult to monitor soley using ground surveys due of the large scale,
distribution and a varity of other difficulties. Remote sensing imagery
provides a solution for these problems encountered with ground based methods.
High resolution remote sensing imagery allows for the detection of informal
settlements exclusively based on image data.
The remote sensing imagery could be annotated manually although this is quite
tedious on the scale of metropolitan areas. As a solution, the detection of informal
settlements is automated using inherent visual characteristics of an area. These features extracted from the remote sensing
imagery should provide a distinct difference between formal and informal
settlements on which classification can be applied. 
