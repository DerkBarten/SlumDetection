\section{Introduction}

The increase in informal settlements is a common issue faced by nations
globally.  Although informal settlements, such as slums, are commonplace in
underdeveloped countries, it affects the developed world as well, for example
refugee camps. Informal settlements might be difficult to monitor merely on the
usage of ground surveys due to the scale, distribution or a varity of other
difficulties. Remote sensing imagery provides a solution to the problems
encountered with ground based methods.  High resolution remote sensing imagery
allows for the detection of informal settlements exclusively based on features
extracted from the image.  The remote sensing imagery could conceivably be
annotated manually although the feasability diminishes with an increased scale.
As a solution, the detection of informal settlements is automated using
inherent visual characteristics of a certain area. The features extracted from remote sensing imagery provide a distinct difference between formal and
informal settlements which can be used for machine learning classification
algorithms.
