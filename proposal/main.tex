\documentclass{article}
\usepackage[utf8]{inputenc}
\usepackage[backend=biber]{biblatex}
\usepackage[margin=1.4in]{geometry}

\title{Project Proposal}
\author{Derk Barten}
\date{April 2018}

\addbibresource{references.bib}


\begin{document}

\maketitle

\section{Introduction}


why remote sensing is used
  Remote sensing can give a hint to the development of slums in space-time

description of field of study
can be split into two parts, object bases (OBAI); homogeneous urban patches (HUP)

techniques used

VHR sattelites enable sensing on this scale

\section{Literature Review}

There have been quite a few studies concerning the detection of informal
settlements using remote sensing in the last decade \cite{kuffer2016slums}.
Informal settlements inside and in the proximity of cities have been studied
from all around the world using remote sensing imagery, for instance Colombo
\cite{colombo}, Johannesburg \cite{williams2016automatic}, Accra \cite{accra},
Mumbai \cite{mumbai}, and Hyderabad \cite{hyderabad}. There are numerous
characteristics that can be used to differentiate between formal and informal
settlements. This could be the number vegetation, the width of the roads, the
size and orientations of dwellings \cite{owen2013approach}.

In general, there is a shift towards the usage of objects to classify geographical
features, also known as Geographic Object-based imaga analysis or GEOBIA
\cite{hay2008geographic}.Due to the increase in availability of remote sensing
high resolution imagery, it became possible to distinguish individual objects
on a large scale.  Traditionaly, objects would be as smaller than the pixels
itself, therefore pixels using objects instead of pixels would have ammounted
to the same thing.  The increase in spatial resolution allows for the
interpretation of geographical imagery with individual objects at its basis.
This would enable the characterization of regions based on the type of objects
that inhabit it.  As and example, a study classified rooftop objects from
aerial imagery of an informal settlements in Johannesburg to gather demographic
information as an alternative to traditional survey
\cite{williams2016automatic}.

% pe

When classifying entire neighborhoods, most studies tend to use a combination
of characteristic that are used in the classification between formal and
informal \cite{graesser2012image}. As previously mentioned, the orientation of
buildings is used as a way of discriminating between formal and informal
settlements. A technique used to qualify the orientation is the aptly named Histogram of
Oriented Gradients.



% talk about what feature extraction to use

\subsection{Pixel-based image analysis} Object based does not necessarily be
better than pixel based 

\subsubsection*{Slums from Space—15 Years of Slum Mapping Using Remote Sensing}
\cite{blaschke2014geographic}
\cite{bian2007object}

\cite{kuffer2016slums} is a comprehensive study

\section{Research Question}

\nocite{*}
\printbibliography
\end{document}

potential sources:


Method of and apparatus for pattern recognition (HoG) 

Object-based detailed vegetation classification with airborne high spatial
resolution remote sensing imagery

Automated classification of landform elements using object-based image analysis


############################ related research ##########################

Evaluating the Relationship between Spatial and 
Spectral Features Derived from High Spatial 
Resolution Satellite Data and Urban Poverty in 
Colombo, Sri Lanka

Image based characterization of formal and informal neighborhoods in an urban landscape

An approach to differentiate informal settlements using spectral, texture, geomorphology and road accessibility metrics

The physical face of slums: a structural comparison of slums in Mumbai, India, based on remotely sensed data

Mapping slums using spatial features in Accra, Ghana

Extraction of slum areas from VHR imagery using GLCM variance


The development of a morphological 
unplanned 
settlement index
using very
-
high
-
resolution (VHR) imagery






##################################### graveyard ########################

Geographic
Object-based image analysis (GEOBAI) detect individual objects in a region to
generate certain characteristics that the region posesses.


There are studies that couple poverty metrics with sattlite imaging
\cite{colombo}

