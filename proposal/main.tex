\documentclass{article}
\usepackage[utf8]{inputenc}
\usepackage[backend=biber]{biblatex}
\usepackage[margin=1.4in]{geometry}
\usepackage{pgfgantt}
\usepackage{rotating}


\title{Project Proposal}
\author{Derk Barten}
\date{April 2018}

\addbibresource{references.bib}


\begin{document}

\maketitle

\section{Introduction}


Growth of informal settlements is a common issue faced by nations globally.
These settlements are not limited to slums in undeveloped countries but also
include refugee camps in the developed world. Informal settlements might be
difficult to monitor soley using ground surveys due of the large scale,
distribution and a varity of other difficulties. Remote sensing imagery
provides a solution for these problems encountered with ground based methods.
High resolution remote sensing imagery allows for the detection of informal
settlements exclusively based on image data.
The remote sensing imagery could be annotated manually although this is quite
tedious on the scale of metropolitan areas. As a solution, the detection of informal
settlements is automated using inherent visual characteristics of an area. These features extracted from the remote sensing
imagery should provide a distinct difference between formal and informal
settlements on which classification can be applied. 

\section{Literature Review}

There have been quite a few studies concerning the detection of informal
settlements using remote sensing in the last decade \cite{kuffer2016slums}.
Informal settlements inside and in the proximity of cities have been studied
from all around the world using remote sensing imagery, for instance Colombo
\cite{colombo}, Johannesburg \cite{williams2016automatic}, Accra \cite{accra},
Mumbai \cite{mumbai}, and Hyderabad \cite{hyderabad}. There are numerous
characteristics that can be used to differentiate between formal and informal
settlements. This could be the number vegetation, the width of the roads, the
size and orientations of dwellings \cite{owen2013approach}.

In general, there is a shift towards the usage of objects to classify geographical
features, also known as Geographic Object-based imaga analysis or GEOBIA
\cite{hay2008geographic}.Due to the increase in availability of remote sensing
high resolution imagery, it became possible to distinguish individual objects
on a large scale.  This allows for the
interpretation of geographical imagery with individual objects at its basis.
As a result,  this enable the characterization of regions based on the type of objects
that inhabit it.  As and example, a study classified rooftop objects from
aerial imagery of an informal settlements in Johannesburg to gather demographic
information as an alternative to traditional survey
\cite{williams2016automatic}.

% pe

When classifying entire neighborhoods, it is common for studies to use a combination
of characteristic that are used in the classification between formal and
informal \cite{graesser2012image}. As previously mentioned, the orientation of
buildings is used as a way of discriminating between formal and informal
settlements. A technique used to qualify the orientation is the Histogram of
Oriented Gradients (HoG)\cite{kumar2006discriminative}. This algorithm iterates over blocks of pixels in a given
image and calculate the gradient. The gradients are binned into
the bin that most closely represents their orientation. This effectively
translates the infinitely many orientations of the gradients to just a few.
Structured objects, such as the walls, are characterized by peaks in a few
bins because the gradients in the block will be very alike


% talk about what feature extraction to use

% There are a multitude of features that

% \cite{blaschke2014geographic}
% \cite{bian2007object}


\section{Research Question}

Besides HoG, there are numerous other features that can be used to characterize
neighborhoods. This project will use Line Support Regions \cite{}, Line Feature
Distribution \cite{} as additional features for classification in a similar manner
to \cite{graesser2012image}. These conventional features will be expanded with
a new feature concerning the density of road intersections in an area. The
features will be classified with a number of unsupervised learning algorithms.
In constrast to supervised learning methods, unsupervised algorithms do not
require labeled data. Considering the fast amount of available remote sensing
imagery, this would be a welcome property.

The overall objective of the project is the successfull classification and
detection of informal settlements from sattelite image data. This description
can be broken down into two seperate objectives. The first objective is the
assessment of various features that could characterize neigborhoods. This could
be for instance an assesment of generizabilty, accuracy and other measures
concerning the effectiveness of the feature. The second objective concerns the
evaluation of a number of unsupervised learning algorithms. This will be
approached with in a similar manner as the assessment of the features to
discover the method that best suits the problem domain. Both objectives
together, this project would provide a method for classifying neighborhoods
using remote sensing imagery that is trained with unlabeled training data.


\section{Method and Approach}

This project continues using a software package written for detection of slums
of a previous research project \cite{graesser2012image}. The package is written
in python and has most of the features discussed in the paper already
implemented. These features will be reused and extended with an additional
feature. The usability of the features for classification will be analyzed with
measures that are still to be determined. Once the set of features is validated
to provide an acurate characterization of neighborhoods, the project continues
with the analyzation of different unsupervised approaches.

\section{Evaluation}

The supervisor of the project has provided labeled remote sensing imagery that
will be used for validation. The areas detected as informal settlements using
the developed product will be compared side to side with the ground truth. The
difference overlap will be the margin of error. We could use the measures
accuracy, recall and the F1 score to characterize the results.

\newpage
\section{Plan}
\begin{sidewaysfigure}
\begin{ganttchart}[hgrid, vgrid, x unit=0.21cm, time slot
  format={isodate}]{2018-04-01}{2018-7-06}
  \gantttitlecalendar{year, month, week} \\
  \ganttbar{Project Proposal}{2018-04-01}{2018-04-17}\\
  \ganttbar{HoG}{2018-04-16}{2018-04-22}\\
  \ganttbar{LSR}{2018-04-23}{2018-04-27}\\
  \ganttbar{LFD}{2018-05-06}{2018-05-11}\\
  \ganttbar{New feature}{2018-05-12}{2018-05-18}\\ 
  \ganttbar{Learning algorithms}{2018-05-18}{2018-06-08}\\ 
  \ganttbar{Presentation}{2018-05-28}{2018-06-01}\\
  \ganttbar{Write paper}{2018-06-01}{2018-06-29}\\
  \ganttbar{Presentation}{2018-06-22}{2018-06-29}\\
  \ganttbar{Honours Extension}{2018-06-08}{2018-07-6}\\
\end{ganttchart}
\end{sidewaysfigure}

The timetable for the project will be very tight, please refer to the gantt
table in the apendix. To make matters worse, I will be away between the 28th of
april to the 5th of may. Furthermore, I am also required to extend the thesis
with an honors extention. This I intend to write for a large part with during
the writing of the main thesis in the last few weeks together with an
adittional week in the summer vacation. I aim to write the majority of the main
thesis during the analysis of the features and classification agorithms. This
saves a lot of time at the end of the project and allows me to write about the
subjects when I am engaged in the research. The schedule is optimistic but
I think it will be possible with enough effort.

% \nocite{*}
\printbibliography

\section{Appendix}
\end{document}

potential sources:


Method of and apparatus for pattern recognition (HoG) 

Object-based detailed vegetation classification with airborne high spatial
resolution remote sensing imagery

Automated classification of landform elements using object-based image analysis

discriminative random fields

############################ related research ##########################

Evaluating the Relationship between Spatial and 
Spectral Features Derived from High Spatial 
Resolution Satellite Data and Urban Poverty in 
Colombo, Sri Lanka

Image based characterization of formal and informal neighborhoods in an urban landscape

An approach to differentiate informal settlements using spectral, texture, geomorphology and road accessibility metrics

The physical face of slums: a structural comparison of slums in Mumbai, India, based on remotely sensed data

Mapping slums using spatial features in Accra, Ghana

Extraction of slum areas from VHR imagery using GLCM variance


The development of a morphological 
unplanned 
settlement index
using very
-
high
-
resolution (VHR) imagery






##################################### graveyard ########################

Geographic
Object-based image analysis (GEOBAI) detect individual objects in a region to
generate certain characteristics that the region posesses.


There are studies that couple poverty metrics with sattlite imaging
\cite{colombo}

